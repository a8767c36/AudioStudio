\begin{minipage}{0.5\textwidth}
\textit{signal} \dotfill $\mathbb{R} \to \mathbb{C}$  \\
$i^2 = -1$  \\
\end{minipage}

\(
	\text{correlate}\left(  a,\ b \right)_t = a_t \cdot \overline{b_t}
\)

\(
	\text{fourier}\left( \text{out}\ \textit{(signal)},\ \text{in}\ \textit{(signal)},\ \text{frequency} \in \mathbb{R},\ \text{bandwidth} \in \mathbb{R} \right)\ \{  \\
	{}\qquad \text{Let}\ a = \text{in}  \\
	{}\qquad \text{Let}\ b_t = e^{i \cdot \omega \cdot t}  \qquad\text{with}\ \omega =  2 \pi \cdot\text{frequency}  \qquad\forall~t \in \text{range}  \\
	{}\qquad \text{out} \leftarrow \text{correlate}(a, b)  \\
	{}\qquad \text{out} \leftarrow \text{lowPass} (\text{out}, \text{bandwidth})  \\
	{}\qquad \text{out} \leftarrow \text{lowPass'}(\text{out}, \text{bandwidth})  \\
	\}
\)

The beauty of that definition lies within its simplicity: \\
It only consists of building blocks that are simple to implement and cheap in run-time cost. \\
The `lowPass' and `correlate' procedures need only make a single pass over the signal data, thus run in $\mathcal{O}(n)$ time.  \\
$e^{i \cdot \omega \cdot t}$ too can be implemented to run very quickly.  \\
