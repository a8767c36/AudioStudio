To express my thoughts,
I need to introduce you to the concept of a ``standing wave'':
In my mental model, it's an oscillation that, while in constant movement,
conserves a special property.
An example would be the swinging of a pendulum that conserves its energy:
The energy contained within the movement is constant, in contrast to the oscillating properties (position, velocity).
This conserved constant is a characteristic of the oscillation and describes its ``3. axis''.
While we can imagine oscillation like circular movement, we can imagine the conserved constant as an axis:
Combine it and you get a ``Helix''.

This is what the following is about.
When we hear a sound, we don't exactely register the periodic movement of the air pressure,
but rather the ``preserved property'' (which is the sound with its specific frequency/intensity per se).
When we use a language to express something, we construct words with constant meaning out of a stream of varying values (a \textit{signal}).
Language is all about signals: We don't focus on the volatile atoms, but rather at the ``direction at which we're going'': The 3. Axis.

